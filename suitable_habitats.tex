% Options for packages loaded elsewhere
% Options for packages loaded elsewhere
\PassOptionsToPackage{unicode}{hyperref}
\PassOptionsToPackage{hyphens}{url}
\PassOptionsToPackage{dvipsnames,svgnames,x11names}{xcolor}
%
\documentclass[
  letterpaper,
  DIV=11,
  numbers=noendperiod]{scrartcl}
\usepackage{xcolor}
\usepackage{amsmath,amssymb}
\setcounter{secnumdepth}{-\maxdimen} % remove section numbering
\usepackage{iftex}
\ifPDFTeX
  \usepackage[T1]{fontenc}
  \usepackage[utf8]{inputenc}
  \usepackage{textcomp} % provide euro and other symbols
\else % if luatex or xetex
  \usepackage{unicode-math} % this also loads fontspec
  \defaultfontfeatures{Scale=MatchLowercase}
  \defaultfontfeatures[\rmfamily]{Ligatures=TeX,Scale=1}
\fi
\usepackage{lmodern}
\ifPDFTeX\else
  % xetex/luatex font selection
\fi
% Use upquote if available, for straight quotes in verbatim environments
\IfFileExists{upquote.sty}{\usepackage{upquote}}{}
\IfFileExists{microtype.sty}{% use microtype if available
  \usepackage[]{microtype}
  \UseMicrotypeSet[protrusion]{basicmath} % disable protrusion for tt fonts
}{}
\makeatletter
\@ifundefined{KOMAClassName}{% if non-KOMA class
  \IfFileExists{parskip.sty}{%
    \usepackage{parskip}
  }{% else
    \setlength{\parindent}{0pt}
    \setlength{\parskip}{6pt plus 2pt minus 1pt}}
}{% if KOMA class
  \KOMAoptions{parskip=half}}
\makeatother
% Make \paragraph and \subparagraph free-standing
\makeatletter
\ifx\paragraph\undefined\else
  \let\oldparagraph\paragraph
  \renewcommand{\paragraph}{
    \@ifstar
      \xxxParagraphStar
      \xxxParagraphNoStar
  }
  \newcommand{\xxxParagraphStar}[1]{\oldparagraph*{#1}\mbox{}}
  \newcommand{\xxxParagraphNoStar}[1]{\oldparagraph{#1}\mbox{}}
\fi
\ifx\subparagraph\undefined\else
  \let\oldsubparagraph\subparagraph
  \renewcommand{\subparagraph}{
    \@ifstar
      \xxxSubParagraphStar
      \xxxSubParagraphNoStar
  }
  \newcommand{\xxxSubParagraphStar}[1]{\oldsubparagraph*{#1}\mbox{}}
  \newcommand{\xxxSubParagraphNoStar}[1]{\oldsubparagraph{#1}\mbox{}}
\fi
\makeatother

\usepackage{color}
\usepackage{fancyvrb}
\newcommand{\VerbBar}{|}
\newcommand{\VERB}{\Verb[commandchars=\\\{\}]}
\DefineVerbatimEnvironment{Highlighting}{Verbatim}{commandchars=\\\{\}}
% Add ',fontsize=\small' for more characters per line
\usepackage{framed}
\definecolor{shadecolor}{RGB}{241,243,245}
\newenvironment{Shaded}{\begin{snugshade}}{\end{snugshade}}
\newcommand{\AlertTok}[1]{\textcolor[rgb]{0.68,0.00,0.00}{#1}}
\newcommand{\AnnotationTok}[1]{\textcolor[rgb]{0.37,0.37,0.37}{#1}}
\newcommand{\AttributeTok}[1]{\textcolor[rgb]{0.40,0.45,0.13}{#1}}
\newcommand{\BaseNTok}[1]{\textcolor[rgb]{0.68,0.00,0.00}{#1}}
\newcommand{\BuiltInTok}[1]{\textcolor[rgb]{0.00,0.23,0.31}{#1}}
\newcommand{\CharTok}[1]{\textcolor[rgb]{0.13,0.47,0.30}{#1}}
\newcommand{\CommentTok}[1]{\textcolor[rgb]{0.37,0.37,0.37}{#1}}
\newcommand{\CommentVarTok}[1]{\textcolor[rgb]{0.37,0.37,0.37}{\textit{#1}}}
\newcommand{\ConstantTok}[1]{\textcolor[rgb]{0.56,0.35,0.01}{#1}}
\newcommand{\ControlFlowTok}[1]{\textcolor[rgb]{0.00,0.23,0.31}{\textbf{#1}}}
\newcommand{\DataTypeTok}[1]{\textcolor[rgb]{0.68,0.00,0.00}{#1}}
\newcommand{\DecValTok}[1]{\textcolor[rgb]{0.68,0.00,0.00}{#1}}
\newcommand{\DocumentationTok}[1]{\textcolor[rgb]{0.37,0.37,0.37}{\textit{#1}}}
\newcommand{\ErrorTok}[1]{\textcolor[rgb]{0.68,0.00,0.00}{#1}}
\newcommand{\ExtensionTok}[1]{\textcolor[rgb]{0.00,0.23,0.31}{#1}}
\newcommand{\FloatTok}[1]{\textcolor[rgb]{0.68,0.00,0.00}{#1}}
\newcommand{\FunctionTok}[1]{\textcolor[rgb]{0.28,0.35,0.67}{#1}}
\newcommand{\ImportTok}[1]{\textcolor[rgb]{0.00,0.46,0.62}{#1}}
\newcommand{\InformationTok}[1]{\textcolor[rgb]{0.37,0.37,0.37}{#1}}
\newcommand{\KeywordTok}[1]{\textcolor[rgb]{0.00,0.23,0.31}{\textbf{#1}}}
\newcommand{\NormalTok}[1]{\textcolor[rgb]{0.00,0.23,0.31}{#1}}
\newcommand{\OperatorTok}[1]{\textcolor[rgb]{0.37,0.37,0.37}{#1}}
\newcommand{\OtherTok}[1]{\textcolor[rgb]{0.00,0.23,0.31}{#1}}
\newcommand{\PreprocessorTok}[1]{\textcolor[rgb]{0.68,0.00,0.00}{#1}}
\newcommand{\RegionMarkerTok}[1]{\textcolor[rgb]{0.00,0.23,0.31}{#1}}
\newcommand{\SpecialCharTok}[1]{\textcolor[rgb]{0.37,0.37,0.37}{#1}}
\newcommand{\SpecialStringTok}[1]{\textcolor[rgb]{0.13,0.47,0.30}{#1}}
\newcommand{\StringTok}[1]{\textcolor[rgb]{0.13,0.47,0.30}{#1}}
\newcommand{\VariableTok}[1]{\textcolor[rgb]{0.07,0.07,0.07}{#1}}
\newcommand{\VerbatimStringTok}[1]{\textcolor[rgb]{0.13,0.47,0.30}{#1}}
\newcommand{\WarningTok}[1]{\textcolor[rgb]{0.37,0.37,0.37}{\textit{#1}}}

\usepackage{longtable,booktabs,array}
\usepackage{calc} % for calculating minipage widths
% Correct order of tables after \paragraph or \subparagraph
\usepackage{etoolbox}
\makeatletter
\patchcmd\longtable{\par}{\if@noskipsec\mbox{}\fi\par}{}{}
\makeatother
% Allow footnotes in longtable head/foot
\IfFileExists{footnotehyper.sty}{\usepackage{footnotehyper}}{\usepackage{footnote}}
\makesavenoteenv{longtable}
\usepackage{graphicx}
\makeatletter
\newsavebox\pandoc@box
\newcommand*\pandocbounded[1]{% scales image to fit in text height/width
  \sbox\pandoc@box{#1}%
  \Gscale@div\@tempa{\textheight}{\dimexpr\ht\pandoc@box+\dp\pandoc@box\relax}%
  \Gscale@div\@tempb{\linewidth}{\wd\pandoc@box}%
  \ifdim\@tempb\p@<\@tempa\p@\let\@tempa\@tempb\fi% select the smaller of both
  \ifdim\@tempa\p@<\p@\scalebox{\@tempa}{\usebox\pandoc@box}%
  \else\usebox{\pandoc@box}%
  \fi%
}
% Set default figure placement to htbp
\def\fps@figure{htbp}
\makeatother





\setlength{\emergencystretch}{3em} % prevent overfull lines

\providecommand{\tightlist}{%
  \setlength{\itemsep}{0pt}\setlength{\parskip}{0pt}}



 


\usepackage{booktabs}
\usepackage{longtable}
\usepackage{array}
\usepackage{multirow}
\usepackage{wrapfig}
\usepackage{float}
\usepackage{colortbl}
\usepackage{pdflscape}
\usepackage{tabu}
\usepackage{threeparttable}
\usepackage{threeparttablex}
\usepackage[normalem]{ulem}
\usepackage{makecell}
\usepackage{xcolor}
\KOMAoption{captions}{tableheading}
\makeatletter
\@ifpackageloaded{caption}{}{\usepackage{caption}}
\AtBeginDocument{%
\ifdefined\contentsname
  \renewcommand*\contentsname{Table of contents}
\else
  \newcommand\contentsname{Table of contents}
\fi
\ifdefined\listfigurename
  \renewcommand*\listfigurename{List of Figures}
\else
  \newcommand\listfigurename{List of Figures}
\fi
\ifdefined\listtablename
  \renewcommand*\listtablename{List of Tables}
\else
  \newcommand\listtablename{List of Tables}
\fi
\ifdefined\figurename
  \renewcommand*\figurename{Figure}
\else
  \newcommand\figurename{Figure}
\fi
\ifdefined\tablename
  \renewcommand*\tablename{Table}
\else
  \newcommand\tablename{Table}
\fi
}
\@ifpackageloaded{float}{}{\usepackage{float}}
\floatstyle{ruled}
\@ifundefined{c@chapter}{\newfloat{codelisting}{h}{lop}}{\newfloat{codelisting}{h}{lop}[chapter]}
\floatname{codelisting}{Listing}
\newcommand*\listoflistings{\listof{codelisting}{List of Listings}}
\makeatother
\makeatletter
\makeatother
\makeatletter
\@ifpackageloaded{caption}{}{\usepackage{caption}}
\@ifpackageloaded{subcaption}{}{\usepackage{subcaption}}
\makeatother
\usepackage{bookmark}
\IfFileExists{xurl.sty}{\usepackage{xurl}}{} % add URL line breaks if available
\urlstyle{same}
\hypersetup{
  pdftitle={Determining optimal suitability in Exclusive Economic Zones (EEZ) on the West Coast of USA},
  pdfauthor={Leela Dixit},
  colorlinks=true,
  linkcolor={blue},
  filecolor={Maroon},
  citecolor={Blue},
  urlcolor={Blue},
  pdfcreator={LaTeX via pandoc}}


\title{Determining optimal suitability in Exclusive Economic Zones (EEZ)
on the West Coast of USA}
\author{Leela Dixit}
\date{2025-11-29}
\begin{document}
\maketitle


Load packages

\begin{Shaded}
\begin{Highlighting}[]
\FunctionTok{library}\NormalTok{(tidyverse)}
\FunctionTok{library}\NormalTok{(here)}
\FunctionTok{library}\NormalTok{(tmap)}
\FunctionTok{library}\NormalTok{(stars)}
\FunctionTok{library}\NormalTok{(raster)}
\FunctionTok{library}\NormalTok{(terra)}
\end{Highlighting}
\end{Shaded}

Read in data

\begin{Shaded}
\begin{Highlighting}[]
\CommentTok{\# exclusive economic zones for west coast}
\NormalTok{eez }\OtherTok{\textless{}{-}} \FunctionTok{st\_read}\NormalTok{(}\FunctionTok{here}\NormalTok{(}\StringTok{"data"}\NormalTok{, }\StringTok{"wc\_regions\_clean.shp"}\NormalTok{))}

\CommentTok{\# bathymetry}
\NormalTok{depth }\OtherTok{\textless{}{-}} \FunctionTok{rast}\NormalTok{(}\FunctionTok{here}\NormalTok{(}\StringTok{"data"}\NormalTok{, }\StringTok{"depth.tif"}\NormalTok{))}

\CommentTok{\# sea surface temperature}
\CommentTok{\# read in as a stack}
\NormalTok{filelist }\OtherTok{\textless{}{-}} \FunctionTok{list.files}\NormalTok{(}\FunctionTok{here}\NormalTok{(}\StringTok{"data"}\NormalTok{, }\StringTok{"average\_annual\_sst"}\NormalTok{), }\AttributeTok{full.names =} \ConstantTok{TRUE}\NormalTok{)}
\NormalTok{sst }\OtherTok{\textless{}{-}} \FunctionTok{rast}\NormalTok{(filelist)}
\end{Highlighting}
\end{Shaded}

\subsection{Step 1: Prepare the data}\label{step-1-prepare-the-data}

Check the CRS of all data and transform if needed.

\begin{Shaded}
\begin{Highlighting}[]
\CommentTok{\# check crs for all data files}
\FunctionTok{crs}\NormalTok{(sst, }\AttributeTok{describe =} \ConstantTok{TRUE}\NormalTok{)}\SpecialCharTok{$}\NormalTok{code }
\end{Highlighting}
\end{Shaded}

\begin{verbatim}
[1] NA
\end{verbatim}

\begin{Shaded}
\begin{Highlighting}[]
\FunctionTok{st\_crs}\NormalTok{(eez)}\SpecialCharTok{$}\NormalTok{epsg}
\end{Highlighting}
\end{Shaded}

\begin{verbatim}
[1] 4326
\end{verbatim}

\begin{Shaded}
\begin{Highlighting}[]
\FunctionTok{crs}\NormalTok{(depth, }\AttributeTok{describe =} \ConstantTok{TRUE}\NormalTok{)}\SpecialCharTok{$}\NormalTok{code}
\end{Highlighting}
\end{Shaded}

\begin{verbatim}
[1] "4326"
\end{verbatim}

\begin{Shaded}
\begin{Highlighting}[]
\CommentTok{\# transform sea surface temp to 4326}
\FunctionTok{crs}\NormalTok{(sst) }\OtherTok{\textless{}{-}} \StringTok{\textquotesingle{}epsg:4326\textquotesingle{}}
\CommentTok{\# check again}
\ControlFlowTok{if}\NormalTok{ (}\FunctionTok{crs}\NormalTok{(sst, }\AttributeTok{describe =} \ConstantTok{TRUE}\NormalTok{)}\SpecialCharTok{$}\NormalTok{code }\SpecialCharTok{==} \FunctionTok{st\_crs}\NormalTok{(eez)}\SpecialCharTok{$}\NormalTok{epsg) \{}
  \FunctionTok{print}\NormalTok{(}\StringTok{"CRS match!"}\NormalTok{)}
\NormalTok{\} }\ControlFlowTok{else}\NormalTok{ \{}
  \FunctionTok{stop}\NormalTok{(}\StringTok{"CRS do not match, must transform"}\NormalTok{)}
\NormalTok{\}}
\end{Highlighting}
\end{Shaded}

\begin{verbatim}
[1] "CRS match!"
\end{verbatim}

\subsection{Step 2: Process data}\label{step-2-process-data}

Clean sea surface temperature and depth data.

\begin{Shaded}
\begin{Highlighting}[]
\CommentTok{\# look at layer names}
\FunctionTok{names}\NormalTok{(sst)}
\end{Highlighting}
\end{Shaded}

\begin{verbatim}
[1] "average_annual_sst_2008" "average_annual_sst_2009"
[3] "average_annual_sst_2010" "average_annual_sst_2011"
[5] "average_annual_sst_2012"
\end{verbatim}

\begin{Shaded}
\begin{Highlighting}[]
\CommentTok{\# find the mean SST from all years}
\CommentTok{\# sst\_avg \textless{}{-} app(sst, mean, na.rm = TRUE)}
\NormalTok{sst\_avg }\OtherTok{\textless{}{-}} \FunctionTok{mean}\NormalTok{(sst)}
\FunctionTok{names}\NormalTok{(sst\_avg) }\CommentTok{\# just one layer!}
\end{Highlighting}
\end{Shaded}

\begin{verbatim}
[1] "mean"
\end{verbatim}

\begin{Shaded}
\begin{Highlighting}[]
\CommentTok{\# convert sst from kelvin to celsius}
\NormalTok{sst\_avg }\OtherTok{\textless{}{-}}\NormalTok{ sst\_avg }\SpecialCharTok{{-}} \FloatTok{273.15}

\CommentTok{\# crop depth raster to the extent of sst }
\NormalTok{depth\_crop }\OtherTok{\textless{}{-}} \FunctionTok{crop}\NormalTok{(depth, }\FunctionTok{floor}\NormalTok{(}\FunctionTok{ext}\NormalTok{(sst\_avg)))}
\CommentTok{\# check extents match }
\FunctionTok{ext}\NormalTok{(depth\_crop) }\SpecialCharTok{==} \FunctionTok{ext}\NormalTok{(sst\_avg) }\CommentTok{\# they are very close, but extents don\textquotesingle{}t match}
\end{Highlighting}
\end{Shaded}

\begin{verbatim}
[1] FALSE
\end{verbatim}

\begin{Shaded}
\begin{Highlighting}[]
\CommentTok{\# resample depth data to match resolution to sst data}
\CommentTok{\# use nearest neighbor approach}
\NormalTok{depth\_resample }\OtherTok{\textless{}{-}} \FunctionTok{resample}\NormalTok{(depth\_crop, sst\_avg, }\AttributeTok{method =} \StringTok{"near"}\NormalTok{)}

\CommentTok{\# check for match in resolution, extent, and crs}
\ControlFlowTok{if}\NormalTok{ (}\FunctionTok{res}\NormalTok{(sst\_avg)[}\DecValTok{1}\NormalTok{] }\SpecialCharTok{==} \FunctionTok{res}\NormalTok{(depth\_resample)[}\DecValTok{1}\NormalTok{] }\SpecialCharTok{\&}
    \FunctionTok{res}\NormalTok{(sst\_avg)[}\DecValTok{2}\NormalTok{] }\SpecialCharTok{==} \FunctionTok{res}\NormalTok{(depth\_resample)[}\DecValTok{2}\NormalTok{] }\SpecialCharTok{\&}
    \FunctionTok{ext}\NormalTok{(sst\_avg) }\SpecialCharTok{==} \FunctionTok{ext}\NormalTok{(depth\_resample) }\SpecialCharTok{\&}
    \FunctionTok{crs}\NormalTok{(sst\_avg) }\SpecialCharTok{==} \FunctionTok{crs}\NormalTok{(depth\_resample))\{}
  \FunctionTok{print}\NormalTok{(}\StringTok{"Resolution, extent, and CRS match!"}\NormalTok{)}
\NormalTok{\} }\ControlFlowTok{else}\NormalTok{ \{}
  \FunctionTok{stop}\NormalTok{(}\StringTok{"Resolution, extent, and/or CRS do not match."}\NormalTok{)}
\NormalTok{\}}
\end{Highlighting}
\end{Shaded}

\begin{verbatim}
[1] "Resolution, extent, and CRS match!"
\end{verbatim}

\begin{Shaded}
\begin{Highlighting}[]
\CommentTok{\# combine depth and sst to make sure they match}
\NormalTok{check\_match }\OtherTok{\textless{}{-}} \FunctionTok{c}\NormalTok{(depth\_resample, sst\_avg)}
\end{Highlighting}
\end{Shaded}

\subsection{Step 3: Find suitable
locations}\label{step-3-find-suitable-locations}

Find suitable locations for Oysters.

Sea Surface Temperature : 11-30

Depth : 0-70

\begin{Shaded}
\begin{Highlighting}[]
\CommentTok{\# min max sst and depth are replaceable for a function}
\CommentTok{\# reclassify for suitable conditions}
\NormalTok{rcl\_sst }\OtherTok{\textless{}{-}} \FunctionTok{matrix}\NormalTok{(}\FunctionTok{c}\NormalTok{(}\SpecialCharTok{{-}}\ConstantTok{Inf}\NormalTok{, }\DecValTok{11}\NormalTok{, }\DecValTok{0}\NormalTok{,}
                    \DecValTok{11}\NormalTok{, }\DecValTok{30}\NormalTok{, }\DecValTok{1}\NormalTok{, }
                    \DecValTok{30}\NormalTok{, }\ConstantTok{Inf}\NormalTok{, }\DecValTok{0}\NormalTok{),}
                    \AttributeTok{ncol =} \DecValTok{3}\NormalTok{, }\AttributeTok{byrow =} \ConstantTok{TRUE}\NormalTok{)}

\NormalTok{rcl\_depth }\OtherTok{\textless{}{-}} \FunctionTok{matrix}\NormalTok{(}\FunctionTok{c}\NormalTok{(}\SpecialCharTok{{-}}\ConstantTok{Inf}\NormalTok{, }\SpecialCharTok{{-}}\DecValTok{70}\NormalTok{, }\DecValTok{0}\NormalTok{,}
                      \SpecialCharTok{{-}}\DecValTok{70}\NormalTok{, }\DecValTok{0}\NormalTok{, }\DecValTok{1}\NormalTok{,}
                      \DecValTok{0}\NormalTok{, }\ConstantTok{Inf}\NormalTok{, }\DecValTok{0}\NormalTok{),}
                      \AttributeTok{ncol =} \DecValTok{3}\NormalTok{, }\AttributeTok{byrow =} \ConstantTok{TRUE}\NormalTok{)}

\CommentTok{\# reclassify per layer for each matrix}
\NormalTok{sst\_reclass }\OtherTok{\textless{}{-}} \FunctionTok{classify}\NormalTok{(sst\_avg, rcl\_sst)}
\NormalTok{depth\_reclass }\OtherTok{\textless{}{-}} \FunctionTok{classify}\NormalTok{(depth\_resample, rcl\_depth)}
\CommentTok{\# check min and max are 0 and 1}
\FunctionTok{summary}\NormalTok{(sst\_reclass)}
\end{Highlighting}
\end{Shaded}

\begin{verbatim}
      mean      
 Min.   :0.000  
 1st Qu.:1.000  
 Median :1.000  
 Mean   :0.929  
 3rd Qu.:1.000  
 Max.   :1.000  
 NA's   :42349  
\end{verbatim}

\begin{Shaded}
\begin{Highlighting}[]
\FunctionTok{summary}\NormalTok{(depth\_reclass)}
\end{Highlighting}
\end{Shaded}

\begin{verbatim}
     depth        
 Min.   :0.00000  
 1st Qu.:0.00000  
 Median :0.00000  
 Mean   :0.01353  
 3rd Qu.:0.00000  
 Max.   :1.00000  
\end{verbatim}

\begin{Shaded}
\begin{Highlighting}[]
\CommentTok{\# depth x sst will return only cells with both 1s as 1}
\NormalTok{multiplication\_func }\OtherTok{\textless{}{-}} \ControlFlowTok{function}\NormalTok{(x, y) \{}
\NormalTok{  x }\SpecialCharTok{*}\NormalTok{ y}
\NormalTok{\}}
\NormalTok{suitable\_cells }\OtherTok{\textless{}{-}} \FunctionTok{lapp}\NormalTok{(}\FunctionTok{c}\NormalTok{(sst\_reclass, depth\_reclass), }\AttributeTok{fun =}\NormalTok{ multiplication\_func)}
\CommentTok{\# check min and max are 0 and 1}
\FunctionTok{summary}\NormalTok{(suitable\_cells)}
\end{Highlighting}
\end{Shaded}

\begin{verbatim}
      lyr1      
 Min.   :0.000  
 1st Qu.:0.000  
 Median :0.000  
 Mean   :0.008  
 3rd Qu.:0.000  
 Max.   :1.000  
 NA's   :42349  
\end{verbatim}

\begin{Shaded}
\begin{Highlighting}[]
\CommentTok{\# create mask with EEZ so nothing too deep in the ocean and no on{-}land is returned}
\NormalTok{suitable\_cells\_west }\OtherTok{\textless{}{-}} \FunctionTok{mask}\NormalTok{(suitable\_cells, eez)}
\FunctionTok{summary}\NormalTok{(suitable\_cells\_west)}
\end{Highlighting}
\end{Shaded}

\begin{verbatim}
      lyr1      
 Min.   :0.000  
 1st Qu.:0.000  
 Median :0.000  
 Mean   :0.014  
 3rd Qu.:0.000  
 Max.   :1.000  
 NA's   :74795  
\end{verbatim}

\begin{Shaded}
\begin{Highlighting}[]
\CommentTok{\# quick plot to check mask worked}
\FunctionTok{plot}\NormalTok{(suitable\_cells)}
\end{Highlighting}
\end{Shaded}

\pandocbounded{\includegraphics[keepaspectratio]{suitable_habitats_files/figure-pdf/unnamed-chunk-5-1.pdf}}

\begin{Shaded}
\begin{Highlighting}[]
\FunctionTok{plot}\NormalTok{(suitable\_cells\_west)}
\end{Highlighting}
\end{Shaded}

\pandocbounded{\includegraphics[keepaspectratio]{suitable_habitats_files/figure-pdf/unnamed-chunk-5-2.pdf}}

\begin{Shaded}
\begin{Highlighting}[]
\CommentTok{\# how many cells are suitable in each zone}
\NormalTok{suitable\_cells\_west[suitable\_cells\_west }\SpecialCharTok{==} \DecValTok{0}\NormalTok{] }\OtherTok{\textless{}{-}} \ConstantTok{NA}
\NormalTok{eez\_rast }\OtherTok{\textless{}{-}} \FunctionTok{rasterize}\NormalTok{(eez, }\AttributeTok{y =}\NormalTok{ suitable\_cells\_west, }\AttributeTok{field =} \StringTok{"rgn\_id"}\NormalTok{)}
\NormalTok{suitable\_cells\_west\_zone }\OtherTok{\textless{}{-}} \FunctionTok{zonal}\NormalTok{(suitable\_cells\_west, eez\_rast, }\AttributeTok{fun =} \StringTok{"notNA"}\NormalTok{)}
\NormalTok{knitr}\SpecialCharTok{::}\FunctionTok{kable}\NormalTok{(suitable\_cells\_west\_zone, }\AttributeTok{col.names =} \FunctionTok{c}\NormalTok{(}\StringTok{"Region ID"}\NormalTok{, }\StringTok{"Suitable Cell Count"}\NormalTok{), }\AttributeTok{table.attr =} \StringTok{\textquotesingle{}data{-}quarto{-}disable{-}processing="true"\textquotesingle{}}\NormalTok{) }\SpecialCharTok{\%\textgreater{}\%} 
\NormalTok{  kableExtra}\SpecialCharTok{::}\FunctionTok{kable\_styling}\NormalTok{(}\AttributeTok{bootstrap\_options =} \FunctionTok{c}\NormalTok{(}\StringTok{"basic"}\NormalTok{, }\StringTok{"hover"}\NormalTok{), }\AttributeTok{full\_width =} \ConstantTok{FALSE}\NormalTok{)}
\end{Highlighting}
\end{Shaded}

\begin{longtable*}[t]{rr}
\toprule
Region ID & Suitable Cell Count\\
\midrule
1 & 71\\
2 & 11\\
3 & 238\\
4 & 211\\
5 & 162\\
\bottomrule
\end{longtable*}

\begin{Shaded}
\begin{Highlighting}[]
\CommentTok{\# find cell size of suitable cell areas}
\NormalTok{suitable\_cell\_area }\OtherTok{\textless{}{-}} \FunctionTok{cellSize}\NormalTok{(suitable\_cells\_west, }\AttributeTok{unit=}\StringTok{"km"}\NormalTok{)}
\FunctionTok{summary}\NormalTok{(suitable\_cell\_area)}
\end{Highlighting}
\end{Shaded}

\begin{verbatim}
      area      
 Min.   :13.85  
 1st Qu.:15.20  
 Median :16.45  
 Mean   :16.37  
 3rd Qu.:17.58  
 Max.   :18.56  
\end{verbatim}

\begin{Shaded}
\begin{Highlighting}[]
\CommentTok{\# sum across zones to get total suitable area}
\NormalTok{total\_area }\OtherTok{\textless{}{-}} \FunctionTok{zonal}\NormalTok{(suitable\_cell\_area, eez\_rast, }\AttributeTok{fun=}\StringTok{"sum"}\NormalTok{, }\AttributeTok{na.rm =} \ConstantTok{TRUE}\NormalTok{)}
\NormalTok{knitr}\SpecialCharTok{::}\FunctionTok{kable}\NormalTok{(total\_area, }
             \AttributeTok{col.names =} \FunctionTok{c}\NormalTok{(}\StringTok{"Region ID"}\NormalTok{, }\StringTok{"Total Suitable Area (km)"}\NormalTok{), }\AttributeTok{table.attr =} \StringTok{\textquotesingle{}data{-}quarto{-}disable{-}processing="true"\textquotesingle{}}\NormalTok{) }\SpecialCharTok{\%\textgreater{}\%} 
\NormalTok{  kableExtra}\SpecialCharTok{::}\FunctionTok{kable\_styling}\NormalTok{(}\AttributeTok{bootstrap\_options =} \FunctionTok{c}\NormalTok{(}\StringTok{"basic"}\NormalTok{, }\StringTok{"hover"}\NormalTok{), }
                            \AttributeTok{full\_width =} \ConstantTok{FALSE}\NormalTok{)}
\end{Highlighting}
\end{Shaded}

\begin{longtable*}[t]{rr}
\toprule
Region ID & Total Suitable Area (km)\\
\midrule
1 & 179866.42\\
2 & 163715.00\\
3 & 202779.85\\
4 & 206535.86\\
5 & 67813.69\\
\bottomrule
\end{longtable*}

\begin{Shaded}
\begin{Highlighting}[]
\CommentTok{\# join to spatial data to map}
\NormalTok{total\_area }\OtherTok{\textless{}{-}} \FunctionTok{left\_join}\NormalTok{(total\_area, eez, }\AttributeTok{by =} \StringTok{"rgn\_id"}\NormalTok{)}
\NormalTok{total\_area }\OtherTok{\textless{}{-}} \FunctionTok{st\_as\_sf}\NormalTok{(total\_area)}

\CommentTok{\# map total area in each suitable zone}
\FunctionTok{tm\_shape}\NormalTok{(total\_area) }\SpecialCharTok{+}
  \FunctionTok{tm\_basemap}\NormalTok{(}\StringTok{"CartoDB.PositronNoLabels"}\NormalTok{) }\SpecialCharTok{+}
  \FunctionTok{tm\_polygons}\NormalTok{(}\AttributeTok{fill =} \StringTok{"area"}\NormalTok{,}
              \AttributeTok{breaks =} \FunctionTok{c}\NormalTok{(}\DecValTok{60000}\NormalTok{, }\DecValTok{160000}\NormalTok{, }\DecValTok{170000}\NormalTok{, }\DecValTok{202000}\NormalTok{, }\DecValTok{206000}\NormalTok{, }\ConstantTok{Inf}\NormalTok{),}
              \AttributeTok{title =} \StringTok{"Suitable Area (km)"}\NormalTok{,}
              \AttributeTok{palette =} \StringTok{"{-}viridis"}\NormalTok{) }\SpecialCharTok{+}
  \FunctionTok{tm\_scalebar}\NormalTok{(}\AttributeTok{breaks =} \FunctionTok{c}\NormalTok{(}\DecValTok{0}\NormalTok{, }\DecValTok{100}\NormalTok{, }\DecValTok{200}\NormalTok{, }\DecValTok{300}\NormalTok{),}
              \AttributeTok{position =} \FunctionTok{c}\NormalTok{(}\StringTok{"left"}\NormalTok{, }\StringTok{"bottom"}\NormalTok{)) }\SpecialCharTok{+}
  \FunctionTok{tm\_compass}\NormalTok{(}\AttributeTok{position =} \FunctionTok{c}\NormalTok{(}\StringTok{"left"}\NormalTok{, }\StringTok{"bottom"}\NormalTok{)) }\SpecialCharTok{+}
  \FunctionTok{tm\_title}\NormalTok{(}\StringTok{"West Coast Suitable Area : Oysters"}\NormalTok{)}
\end{Highlighting}
\end{Shaded}

\pandocbounded{\includegraphics[keepaspectratio]{suitable_habitats_files/figure-pdf/unnamed-chunk-5-3.pdf}}

\section{Step 4: Create a reproducible
workflow}\label{step-4-create-a-reproducible-workflow}

Red Abalone are a popular seafood delicacy and were farmed along the
west coast of California prior to the fisheries closure 2018 following
multiple ecological stressors leading to population declines. There are
some farm operations for abalone on the west coast, and multiple
restoration efforts in practice.

We will compile the workflow above into a function to recreate a figure
for Red Abalone.

Sea Surface Temperature : 8 - 18

Depth : 0 - 24

\begin{Shaded}
\begin{Highlighting}[]
\NormalTok{suitable\_location }\OtherTok{\textless{}{-}} \ControlFlowTok{function}\NormalTok{(sst\_low, sst\_high, depth\_low, depth\_high, species\_name) \{}
  \CommentTok{\# set up reclassification matrix}
\NormalTok{  rcl\_sst }\OtherTok{\textless{}{-}} \FunctionTok{matrix}\NormalTok{(}\FunctionTok{c}\NormalTok{(}\SpecialCharTok{{-}}\ConstantTok{Inf}\NormalTok{, sst\_low, }\DecValTok{0}\NormalTok{,}
\NormalTok{                    sst\_low, sst\_high, }\DecValTok{1}\NormalTok{, }
\NormalTok{                    sst\_high, }\ConstantTok{Inf}\NormalTok{, }\DecValTok{0}\NormalTok{),}
                    \AttributeTok{ncol =} \DecValTok{3}\NormalTok{, }\AttributeTok{byrow =} \ConstantTok{TRUE}\NormalTok{)}

\NormalTok{  rcl\_depth }\OtherTok{\textless{}{-}} \FunctionTok{matrix}\NormalTok{(}\FunctionTok{c}\NormalTok{(}\SpecialCharTok{{-}}\ConstantTok{Inf}\NormalTok{, depth\_low, }\DecValTok{0}\NormalTok{,}
\NormalTok{                      depth\_low, depth\_high, }\DecValTok{1}\NormalTok{,}
\NormalTok{                      depth\_high, }\ConstantTok{Inf}\NormalTok{, }\DecValTok{0}\NormalTok{),}
                      \AttributeTok{ncol =} \DecValTok{3}\NormalTok{, }\AttributeTok{byrow =} \ConstantTok{TRUE}\NormalTok{)}
  
  \CommentTok{\# reclassify depth and sst data}
\NormalTok{  sst\_reclass }\OtherTok{\textless{}{-}} \FunctionTok{classify}\NormalTok{(sst\_avg, rcl\_sst)}
\NormalTok{  depth\_reclass }\OtherTok{\textless{}{-}} \FunctionTok{classify}\NormalTok{(depth\_resample, rcl\_depth)}
  
  \CommentTok{\# multiply reclassified layers to return cells that are suitable for our species}
\NormalTok{  multiplication\_func }\OtherTok{\textless{}{-}} \ControlFlowTok{function}\NormalTok{(x, y) \{}
\NormalTok{  x }\SpecialCharTok{*}\NormalTok{ y}
\NormalTok{  \}}
\NormalTok{  suitable\_cells }\OtherTok{\textless{}{-}} \FunctionTok{lapp}\NormalTok{(}\FunctionTok{c}\NormalTok{(sst\_reclass, depth\_reclass), }\AttributeTok{fun =}\NormalTok{ multiplication\_func)}
  \FunctionTok{summary}\NormalTok{(suitable\_cells)}
  
  \CommentTok{\# create mask with EEZ so nothing too deep in the ocean and no on{-}land is returned}
\NormalTok{  suitable\_cells\_west }\OtherTok{\textless{}{-}} \FunctionTok{mask}\NormalTok{(suitable\_cells, eez)}
  \CommentTok{\# set 0 cells to NA}
\NormalTok{  suitable\_cells\_west[suitable\_cells\_west }\SpecialCharTok{==} \DecValTok{0}\NormalTok{] }\OtherTok{\textless{}{-}} \ConstantTok{NA}
  
  \CommentTok{\# find cell size of suitable cell areas}
\NormalTok{  eez\_rast }\OtherTok{\textless{}{-}} \FunctionTok{rasterize}\NormalTok{(eez, }\AttributeTok{y =}\NormalTok{ suitable\_cells\_west, }\AttributeTok{field =} \StringTok{"rgn\_id"}\NormalTok{)}
\NormalTok{  suitable\_cell\_area }\OtherTok{\textless{}{-}} \FunctionTok{cellSize}\NormalTok{(suitable\_cells\_west, }\AttributeTok{unit=}\StringTok{"km"}\NormalTok{)}
  
  \CommentTok{\# sum across zones to get total suitable area, and join with spatial data to map}
\NormalTok{  total\_area }\OtherTok{\textless{}{-}} \FunctionTok{zonal}\NormalTok{(suitable\_cell\_area, eez\_rast, }\AttributeTok{fun=}\StringTok{"sum"}\NormalTok{, }\AttributeTok{na.rm =} \ConstantTok{TRUE}\NormalTok{)}
\NormalTok{  total\_area }\OtherTok{\textless{}{-}} \FunctionTok{left\_join}\NormalTok{(total\_area, eez, }\AttributeTok{by =} \StringTok{"rgn\_id"}\NormalTok{)}
\NormalTok{  total\_area }\OtherTok{\textless{}{-}} \FunctionTok{st\_as\_sf}\NormalTok{(total\_area)}
  
  \CommentTok{\# map suitable areas}
  \FunctionTok{tm\_shape}\NormalTok{(total\_area) }\SpecialCharTok{+}
  \FunctionTok{tm\_basemap}\NormalTok{(}\StringTok{"CartoDB.PositronNoLabels"}\NormalTok{) }\SpecialCharTok{+}
  \FunctionTok{tm\_polygons}\NormalTok{(}\AttributeTok{fill =} \StringTok{"area"}\NormalTok{,}
              \AttributeTok{breaks =} \FunctionTok{c}\NormalTok{(}\DecValTok{60000}\NormalTok{, }\DecValTok{160000}\NormalTok{, }\DecValTok{170000}\NormalTok{, }\DecValTok{202000}\NormalTok{, }\DecValTok{206000}\NormalTok{, }\ConstantTok{Inf}\NormalTok{),}
              \AttributeTok{title =} \StringTok{"Suitable Area (km)"}\NormalTok{,}
              \AttributeTok{palette =} \StringTok{"{-}viridis"}\NormalTok{) }\SpecialCharTok{+}
  \FunctionTok{tm\_scalebar}\NormalTok{(}\AttributeTok{breaks =} \FunctionTok{c}\NormalTok{(}\DecValTok{0}\NormalTok{, }\DecValTok{100}\NormalTok{, }\DecValTok{200}\NormalTok{, }\DecValTok{300}\NormalTok{),}
              \AttributeTok{position =} \FunctionTok{c}\NormalTok{(}\StringTok{"left"}\NormalTok{, }\StringTok{"bottom"}\NormalTok{)) }\SpecialCharTok{+}
  \FunctionTok{tm\_compass}\NormalTok{(}\AttributeTok{position =} \FunctionTok{c}\NormalTok{(}\StringTok{"left"}\NormalTok{, }\StringTok{"bottom"}\NormalTok{)) }\SpecialCharTok{+}
  \FunctionTok{tm\_title}\NormalTok{(}\FunctionTok{paste}\NormalTok{(}\StringTok{"West Coast Suitable Area :"}\NormalTok{, species\_name))}
\NormalTok{\}}


\CommentTok{\# Run the function with suitable conditions for red abalone}
\FunctionTok{suitable\_location}\NormalTok{(}\AttributeTok{sst\_low =} \DecValTok{8}\NormalTok{, }
                  \AttributeTok{sst\_high =} \DecValTok{18}\NormalTok{, }
                  \AttributeTok{depth\_low =} \SpecialCharTok{{-}}\DecValTok{24}\NormalTok{, }
                  \AttributeTok{depth\_high =} \DecValTok{0}\NormalTok{, }
                  \AttributeTok{species\_name =} \StringTok{"Red Abalone"}\NormalTok{)}
\end{Highlighting}
\end{Shaded}

\pandocbounded{\includegraphics[keepaspectratio]{suitable_habitats_files/figure-pdf/unnamed-chunk-6-1.pdf}}




\end{document}
